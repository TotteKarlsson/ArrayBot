\documentclass[11pt,fleqn]{book} % Default font size and left-justified equations

\input{structure} % Insert the commands.tex file which contains the majority of the structure behind the template

\usepackage[toc,page]{appendix}
\usepackage{setspace}

\usepackage{graphicx}
\usepackage{wrapfig}
\usepackage{lscape}
\usepackage{rotating}
\usepackage{epstopdf}


\begin{document}
\newcommand{\ab}{ArrayBot}
\newcommand{\ac}{ArrayCam}
\newcommand{\abc}{ArduinoController}
\newcommand{\ar}{Arduino}
\newcommand{\tl}{ThorLabs}

%----------------------------------------------------------------------------------------
%	TITLE PAGE
%----------------------------------------------------------------------------------------

\begingroup
\thispagestyle{empty}
\begin{tikzpicture}[remember picture,overlay]
\coordinate [below=12cm] (midpoint) at (current page.north);
\node at (current page.north west){
\begin{tikzpicture}[remember picture,overlay]
\node[anchor=north west,inner sep=0pt] at (0,0) {\includegraphics[width=\paperwidth]{background}}; % Background image
\draw[anchor=north] (midpoint) node [fill=ocre!30!white,fill opacity=0.6,text opacity=1,inner sep=1cm]{\Huge\centering\bfseries\sffamily\parbox[c][][t]{\paperwidth}{\centering ArrayBot\\[15pt] % Book title
{\Large Smith Lab, Allen Institute - 2016 - 2019}\\[20pt] % Subtitle
{\huge }}}; % Author name
\end{tikzpicture}};
\end{tikzpicture}
\vfill
\endgroup

%%----------------------------------------------------------------------------------------
%%	COPYRIGHT PAGE
%%----------------------------------------------------------------------------------------
%
%\newpage
%~\vfill
%\thispagestyle{empty}
%
%\noindent Copyright \copyright\ 2016 Allen Institute for Brain Science\\ % Copyright notice
%
%\noindent \textsc{Published by Publisher}\\ % Publisher
%
%\noindent \textsc{book-website.com}\\ % URL
%
%\noindent Licensed under the Creative Commons Attribution-NonCommercial 3.0 Unported License (the ``License''). You may not use this file except in compliance with the License. You may obtain a copy of the License at \url{http://creativecommons.org/licenses/by-nc/3.0}. Unless required by applicable law or agreed to in writing, software distributed under the License is distributed on an \textsc{``as is'' basis, without warranties or conditions of any kind}, either express or implied. See the License for the specific language governing permissions and limitations under the License.\\ % License information
%
%%\noindent \textit{First printing, March 2013} % Printing/edition date
%
%%----------------------------------------------------------------------------------------
%%	TABLE OF CONTENTS
%%----------------------------------------------------------------------------------------
%
\usechapterimagefalse % If you don't want to include a chapter image, use this to toggle images off - it can be enabled later with %\usechapterimagetrue
%
%\chapterimage{chapter_head_1.pdf} % Table of contents heading image
%
%\pagestyle{empty} % No headers
%
%\tableofcontents % Print the table of contents itself
%
%\cleardoublepage % Forces the first chapter to start on an odd page so it's on the right
%
%\pagestyle{fancy} % Print headers again
%
%%----------------------------------------------------------------------------------------
%	PART
%----------------------------------------------------------------------------------------

\part{Part One}

%----------------------------------------------------------------------------------------
%	CHAPTER 1
%----------------------------------------------------------------------------------------

\chapterimage{ch1_1.jpg} % Chapter heading image

\chapter{Overview of \ab{} Software Applications}

\section{Introduction}\index{Introduction}

\doublespacing
This document gives an overview of the software that is named \emph{ArrayBot}.

The ArrayBot software is designed to control a set of motorized stages, a camera and other hardware, in order to assist in the process of collecting biological tissue ribbons produced by an ultra-microtome.

The \ab{} software is partitioned into a set of specialized software applications, \ab{}, ArrayCam and the ArduinoController application.

The \ab{} application is designed to interface with the motorized part of the \ab{} hardware. The hardware can be divided into two roughly identical XYZ stages, the \emph{coverslip} stage and the \emph{whisker stage} (picture). 

the \ab{} UI do expose control interfaces to all motor hardware as well as a user interface for programming automated motor sequences. Automation is discussed in a later section.

In addition, the UI do also contain elements for ribbon separation, and ribbon length control.

The \ac{} application expose and interface with the \ab{} camera, as well as certain control of peripheral lights related to the camera vision. The \ac{} application allow the user to take camera snapshots as well as recording movies.

The Arduino controller application purpose is to communicate with a set of Arduino boards, and forward the controls of these to the \ab{} and \ac{} applications.

The \abc{} is designed to act as a server of its connected (Arduino) hardware.

The following section discusses these applications in greater detail.

\clearpage

\section{Arraybot UI's}
\subsection{\ab{} UI}

\begin{sidewaysfigure}[h]
\centering\includegraphics[scale=0.85]{AB_UI_1}
\caption{\ab{} UI}
\end{sidewaysfigure}

\clearpage

\subsection{\ac{} UI}

\begin{sidewaysfigure}[h]
\centering\includegraphics[scale=0.85]{AC_UI_1}
\caption{\ac{} UI}
\end{sidewaysfigure}

\clearpage
\subsection{\abc{} UI}
\begin{sidewaysfigure}[h]
\centering\includegraphics[scale=0.85]{ARDC_UI_1}
\caption{\abc{} UI}
\end{sidewaysfigure}

\clearpage


\section{Arraybot Hardware Setup}

\subsection{Motors}
\subsection{Arduinos}
\subsection{The Camera}


\chapterimage{ch_2.jpg} 


\chapter{Software Design and Software Components}

\section{Rough Software setup - picture}


\chapter{IPC}
\section{Serial Communication}
\section{The socket server and socket clients}

\chapter{Process Sequencer}
The process sequencer component is designed to allow the user to define and execute a sequence of various actions, e.g. moves.

\subsection{Process Sequence}
\subsection{Process	}
\subsection{XML}
\subsection{Triggers}

\chapter{The JoyStick}
\subsection{JoyStickMessageDispatcher}

\chapter{Controlling the motors}
\section{DeviceManager}
\section{APTDevice}
\subsection{APTMotor}
\subsection{LongTravelStage}
\subsection{TCube Stepper Motor}
\subsection{TCube DCServo}
\subsection{Angle Controller}
\section{XYZUnit}


\chapter{The Arduino Server}
The Arduino server class encapsulate Arduino peripherals in the Arraybot system, i.e. the sensor, puffer and the Leica Arduinos. Clients are served with Arduino related messages over a
TCP/IP socket.  

The Arduino server is a descendant of the IPC server class that is implementing all
network functionality.

The Arduino server forwards any messages sent from the arduino board to connected TCP/IP clients.

The purpose of each one of the three Arduinos, the Puffer board, the Sensor board and the Leica Arduino board, are discussed in sections below.

\section{Messages exchanged with the Arduino Server}
This section discusses the messaging protocol for the Arduino Server. The protocol is defined by a few simple text messages (short strings) that can be exchanged with the server. 

Typically a client sends (1) a command to the server, e.g. "ENABLE\_PUFFER", (2) the server receives and handles the message, possibly causing peripheral hardware to be modified, and (3) a response may be sent back to the client. 

The design emphasizes simplicity and so the error checking implemented in the system is held to a minimum. For example, a client sending a request do not typically implement a mechanism to handle a server response notifying if the request was carried out or not.
Instead, this system rely on careful debugging by the user, in reading error and warning log messages.


\subsection{Incoming messages}

Table \ref{tab:ima} lists commands that the \ar{} server accepts.
\begin{table}[h]
\centering
\begin{tabular}{l l  p{180pt}}
\toprule
\textbf{Text Message} & \textbf{Args} & \textbf{Description} \\
\midrule
RESET\_SECTION\_COUNT 	        & none & This message is a request to reset the current section count. \\
ENABLE\_PUFFER                  & none & Request to enable the puffer. \\  
DISABLE\_PUFFER                 & none & Request to disable the puffer. \\  
ENABLE\_AUTO\_PUFF              & none & Enable AutoPuff mechanims. \\  
DISABLE\_AUTO\_PUFF             & none & Disable AutoPuff mechanims. \\  
SET\_DESIRED\_RIBBON\_LENGTH    & none & Set desired ribbon length. Used when the AutoPuff mechanism is enabled. \\  
PUFF                            & none & Request a (manual) puff. \\  
START\_NEW\_RIBBON              & none & Request initiation of a new ribbon. This command will restore the current cut thickness preset and reset the section count. \\  
TURN\_ON\_LED\_LIGHTS				& -		& Turn on LED's \\
TURN\_OFF\_LED\_LIGHTS				& -		& Turn off LED's \\
TURN\_ON\_COAX\_LIGHTS				& -		& Turn on Coax lights \\
TURN\_OFF\_COAX\_LIGHTS			& -		& Turn on Coax lights \\
TOGGLE\_LED\_LIGHT              & none & Toggle LED light on/off. \\  
TOGGLE\_COAX\_LIGHT             & none & Toggle Coax light on/off. \\  

SET\_FRONTLED                   & none & Set Front LED intensity. \\  
SET\_BACKLED                    & none & Set Back LED intensity. \\  
SET\_COAX                       & none & Set Coax LED intensity. \\  
SET\_CUT\_PRESET                & none & Set cut thickness preset. \\  
SET\_DELTAY                     & none & Set mouse delta Y on the Leica UI. \\  
GET\_SENSOR\_ARDUINO\_STATUS       & none & Get general server status. \\  
SENSOR\_CUSTOM\_MESSAGE      		& none &  \\  
GET\_PUFFER\_ARDUINO\_STATUS       & none &  \\  
GET\_SERVER\_STATUS      			& none &  \\  

\bottomrule
\end{tabular}
\caption{Incoming messages handled by the Arduino server}\label{tab:ima}
\end{table}

\newpage

\subsection{Outgoing messages}
\begin{table}[h]
\centering
\begin{tabular}{l l p{160pt}}
\toprule
\textbf{Text Message} & \textbf{Initiated by} & \textbf{Description}\\
\midrule
GET\_READY\_FOR\_ZERO\_CUT\_1 	& Puffer Arduino & This message is sent two counts before a zero cut is to be executed. \\
GET\_READY\_FOR\_ZERO\_CUT\_2 	& Puffer Arduino & This message is sent one counts before a zero cut is to be executed. \\
SET\_ZERO\_CUT\_1 				& Puffer Arduino & This message is sent when a zero cut is to be executed. \\
RESTORE\_FROM\_ZERO\_CUT 		& Puffer Arduino & This message is sent after a zero cut. \\
\bottomrule
\end{tabular}
\caption{Arduino server response messages}\label{tab:asrm}
\end{table}


\newpage

\section{The Puffer and ribbon separation}
Individually cut sections are held together by a thin film of glue. The glue is part of the 'block-phase' and is necessary in order to successfully create a ribbon of consecutive tissue sections. However, the glue may also make a cut section stick to the knife itself.\\
In order to separate a ribbon from the knife, as to give way to a new ribbon, a mechanism involving an air puffer together with a zero thickness cut, allows for automatic separation of a ribbon from the knife.

The puffer mechanics is controlled by an Arduino, and the zero cut is controlled by serial communication between the Leica ultra-microtome and the host computer.


\subsection{Puffer Arduino Communication Protocol}
The Puffer Arduino hardware are connected to the ArrayBot software using a serial (COM) port. This allow messaging between the PC software and  the Arduino Software. 

The Puffer Arduino implements a simple messaging protocol, involving receive/transmit of a few bytes over the serial port connection.

Table \ref{tab:pac} lists the commands that are currently implemented in the Puffer Arduino Software.

\begin{table}[h]
\centering
\begin{tabular}{l l p{160pt} p{160pt}}
\toprule
\textbf{Command} & \textbf{Arguments} & \textbf{Usage} & \textbf{Note}\\
\midrule
S	& 1 or 0	& Enable/Disable simulation of the Hall Sensor. & For debugging purposes. \\
s	& int		& Set simulation speed. The supplied integer denotes the delay, in ms, between sending simulated Hall sensor switch messages. & For debugging purposes. \\
P	& int	& Set cut thickness preset on the Leica. &  This command results in a request to the Leica Arduino to change the cut thickness preset. Valid values are 1-5\\
Y	& int	& Set deltaY for mouse movement on the Leica. &  \\
e	&   & Enable the puffer &  \\
a	&   & Disable the puffer &  \\
p	&   & Request an immediate puff &  \\
d	& int   & Set puff duration in ms &  \\
v	& int   & Set puff valve speed. Valid values are 0-255. &  \\
i	& int   & Request info about current Arduino state. &  \\
\bottomrule
\end{tabular}
\caption{Puffer Arduino commands}\label{tab:pac}
\end{table}

\begin{table}[h]
\centering
\begin{tabular}{l p{160pt} p{80pt}}
\toprule
\textbf{Message} & \textbf{Usage} & \textbf{Note}\\
\midrule
LEICA MESSAGE: "..." 			& Forwarded message from the Leica Arduino & For debugging. \\
HALL\_SENSOR='VALUE' 			& Message indicating the value of the Hall Sensor. VAL = HIGH or LOW & \\
REQUEST\_CUT\_PRESET\_'VALUE' 	& Indicating we are requesting to change the cut thickness preset on the Leica. VAL = 1-5 & \\
CUT\_PRESET\_VALUE IS INVALID 	& Error response &\\
REQUEST\_DELTA\_Y='VALUE'   	& Indicating request to change the delta Y value on the Leica. VAL = any integer & \\
PUFFER\_ENABLED					& Indicate that the puffer was enabled & \\
PUFFER\_DISABLED				& Indicate that the puffer was disabled & \\
EXECUTED\_PUFF 					& Puffer was executed & \\

\bottomrule
\end{tabular}
\caption{Puffer Arduino Client Messages}
\end{table}


\section{Sensors and Lights}
Lights and sensors are read/controlled by a third Arduino - the 'Sensor' Arduino.


Table \ref{tab:sac} lists the commands that are currently implemented in the Sensor Arduino Software.

\begin{table}[h]
\centering
\begin{tabular}{l l p{160pt} p{160pt}}
\toprule
\textbf{Command} & \textbf{Arguments} & \textbf{Usage} & \textbf{Note}\\
\midrule
1		& -	& Button1 Down & Turn on light? \\
2		& -	&  Down &  				\\
3		& -	&  Down & \\
3		& -	&  Down & \\
4		& -	&  Down & \\
5		& -	&  Down & \\
c		& int	&  Set coax drive 0-255 & \\
f		& int	&  Set front LED drive 0-255 & \\
b		& int	&  Set back LED drive 0-255 & \\
i		& 	&  Return information about states & \\

\bottomrule
\end{tabular}
\caption{Sensor Arduino Commands}\label{tab:sac}
\end{table}

\begin{table}[h]
\centering
\begin{tabular}{l l p{160pt} p{160pt}}
\toprule
\textbf{Response} & \textbf{Description}\\
\midrule
BUTTON\_1\_DOWN & Turn off Coax LEDS \\
BUTTON\_2\_DOWN &  \\
BUTTON\_3\_DOWN &  \\
BUTTON\_4\_DOWN &  \\
BUTTON\_5\_DOWN &  \\
COAX\_DRIVE=		&	\\
FRONT\_LED\_DRIVE=		&	\\
BACK\_LED\_DRIVE=		&	\\
PIN\_1\_DOWN			&	\\
DHT22\_DATA=			& \\
DHT22\_ERROR & \\


\bottomrule
\end{tabular}
\caption{Sensor Arduino Responses}\label{tab:sac}
\end{table}

\clearpage

\section{Hardware Setup}
\subsection{Arduino setup}
The figure and tables below discusses the hardware setup of the three Arduinos, the PC and the leica microtome

\begin{figure}[h]
\centering\includegraphics[scale=0.5]{arduino_hw_setup}
\caption{Arduino hardware setup}
\end{figure}




\begin{appendices}
\chapter{Get the source code}

Public Software Repository: $\mathbf{git@github.com:TotteKarlsson/ArrayBot.git}$

\chapter{Software API's}

\section{ArrayBot Software API's}\index{ArrayBot Software API's}
\subsection{abCore}
\subsection{abVCLCore}

\section{ThirdParty libraries}
\subsection{Poco}
\subsection{libcurl}
\subsection{SQLite}
\subsection{tinyxml2}
\subsection{uc480 and uc480\_tools}
\subsection{Dune Scientific's libraries: mtkCommon, mtkMath, mtkIPC and mtkVCL}

\clearpage

\chapter{Arduino wiring}
\begin{table}[h]
\centering
\begin{tabular}{l l p{160pt}}
\toprule
\textbf{Pin \#} & \textbf{Usage} & \textbf{Note}\\
\midrule
0 	& Serial Receive 	& Used for serial communication \\
1 	& Serial Transmit  	& Used for serial communication \\
2 	& Hall Sensor 		&  \\
3 	& Puffer			&  \\
4 	&  &  \\
5 	&  &  \\
6 	&  &  \\
7 	&  &  \\
8 	&  &  \\
9 	&  &  \\
10 	& Serial Receive  & Connection to the Leica Arduino Serial port \\
11 	& Serial Transmit & Connection to the Leica Arduino Serial port \\
12 	&  &  \\
13 	& LED Illuminates when Hall sensor is high &  \\
\bottomrule
\end{tabular}
\caption{Puffer Arduino Pins}\label{tab:pap}
\end{table}



\begin{table}[h]
\centering
\begin{tabular}{l l p{160pt}}
\toprule
\textbf{Pin \#} & \textbf{Usage} & \textbf{Note}\\
\midrule
0 	& Serial Receive 	& Used for serial communication \\
1 	& Serial Transmit  	& Used for serial communication \\

\bottomrule
\end{tabular}
\caption{Sensor Arduino Pins}
\end{table}


\begin{table}[h]
\centering
\begin{tabular}{l l p{160pt}}
\toprule
\textbf{Pin \#} & \textbf{Usage} & \textbf{Note}\\
\midrule
0 	& Serial Receive 	& Used for serial communication to Puffer Arduino\\
1 	& Serial Transmit  	& Used for serial communication to Puffer Arduino\\
2 	& unused &  \\
3 	& unused &  \\
4 	& unused &  \\
5 	& unused &  \\
6 	& unused &  \\
7 	& unused &  \\
8 	& unused &  \\
9 	& unused &  \\
10 	& unused & \\
11 	& unused &\\
12 	& unused &  \\
13 	& unused &  \\
\bottomrule
\end{tabular}
\caption{Leica Arduino Pins}
\end{table}

\clearpage

\end{appendices}




%%----------------------------------------------------------------------------------------
%%	BIBLIOGRAPHY
%%----------------------------------------------------------------------------------------
%\chapter*{Bibliography}
%\addcontentsline{toc}{chapter}{\textcolor{ocre}{Bibliography}}
%\section*{Books}
%\addcontentsline{toc}{section}{Books}
%\printbibliography[heading=bibempty,type=book]
%\section*{Articles}
%\addcontentsline{toc}{section}{Articles}
%\printbibliography[heading=bibempty,type=article]
%
%%----------------------------------------------------------------------------------------
%%	INDEX
%%----------------------------------------------------------------------------------------
%
%\cleardoublepage
%\phantomsection
%\setlength{\columnsep}{0.75cm}
%\addcontentsline{toc}{chapter}{\textcolor{ocre}{Index}}
%\printindex
%
%%----------------------------------------------------------------------------------------
%
\end{document}
